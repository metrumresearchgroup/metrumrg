\documentclass[titlepage]{article}
\usepackage{listings}
\lstMakeShortInline{|}
\usepackage{courier}
%\usepackage{hyperref}
\usepackage[colorlinks,linkcolor=blue,citecolor=blue,urlcolor=blue,breaklinks=true]{hyperref}
\lstset{basicstyle=\ttfamily\small , breaklines}
%\usepackage[margin=2cm]{geometry}
\usepackage[left=3cm,top=3cm,bottom=3cm, right=3cm,includehead,includefoot,landscape]{geometry}
\usepackage{color}
\usepackage{fancyhdr,lastpage}
\pagestyle{fancy}
\rhead{Metrum Research Group, LLC \\ }
\lhead{\includegraphics[scale=2]{logo.png}}
\cfoot{Page \thepage\ of \pageref{LastPage}}
\fancyhfoffset{.25in}
\renewcommand{\headrulewidth}{0.25pt}
\renewcommand{\footrulewidth}{0pt} 
\setlength{\headheight}{23pt}
\renewcommand{\labelitemiii}{$\circ$}
\usepackage{longtable}
\usepackage{amsmath}
\usepackage[T1]{fontenc}
\usepackage[scaled]{helvet}
\renewcommand*\familydefault{\sfdefault}
\usepackage{courier}
\usepackage{graphicx}
\usepackage{tocbibind}
\usepackage[parfill]{parskip}    % Activate to begin paragraphs with an empty line rather than an indent
\usepackage{upgreek}
\usepackage{textpos}
\usepackage{relsize}
\usepackage{upquote}
% Use \begin{landscape} and end{landscape} to rotate text %%%
\usepackage{pdflscape}
\usepackage{textcomp}
\usepackage{float}
\floatplacement{figure}{H}
\floatplacement{table}{H}
\usepackage[printonlyused,nohyperlinks]{acronym}
\def\bflabel#1{{\large#1\ \ \ \ }\hfill}
\usepackage{fixltx2e}
\setlength{\belowcaptionskip}{10pt}





\usepackage{Sweave}

 
\begin{document}
\vspace*{2cm}
\begin{center}
{\Large Parameter Table}\\
~\\
\today\\
~\\
Tim Bergsma\\
\end{center}
\newpage

\section{Purpose}
This script picks up after model.Rnw to process bootstrap results and make a parameter table. It assumes the current working directory is the script directory containing this file.
\subsection{Package}
\begin{Schunk}
\begin{Sinput}
> library(metrumrg)
\end{Sinput}
\end{Schunk}
\section{inputs}
`wikitab' gives us a quick synthesis of `rlog' and the `lookup' of wiki notation in 1005.ctl.
We do some science on the result first, and then some aesthetics for printing in a \LaTeX table.  Table \ref{p1005}.
\begin{Schunk}
\begin{Sinput}
> tab <- wikitab(1005,'../nonmem')
> tab$estimate <- signif(as.numeric(tab$estimate),3)
> tab$tool <- NULL
> tab$run <- NULL
> tab$se <- NULL
> tab
\end{Sinput}
\begin{Soutput}
   parameter                                   description
1     THETA1                       apparent oral clearance
2     THETA2                central volume of distribution
3     THETA3                      absorption rate constant
4     THETA4                  intercompartmental clearance
5     THETA5             peripheral volume of distribution
6     THETA6                      male effect on clearance
7     THETA7                    weight effect on clearance
8   OMEGA1.1      interindividual variability of clearance
9   OMEGA2.1   interindividual clearance-volume covariance
10  OMEGA2.2 interindividual variability of central volume
11  OMEGA3.1       interindividual clearance-Ka covariance
12  OMEGA3.2          interindividual volume-Ka covariance
13  OMEGA3.3             interindividual variability of Ka
14  SIGMA1.1                            proportional error
15  SIGMA2.2                                additive error
                                                                model estimate
1  CL/F (L/h) ~ theta_1 *  theta_6 ^MALE * (WT/70)^theta_7  * e^eta_1       16
2                          V_c /F (L) ~ theta_2 * (WT/70)^1 * e^eta_2       14
3                                     K_a (h^-1 ) ~ theta_3 * e^eta_3        6
4                                                 Q/F (L/h) ~ theta_4       15
5                                                V_p /F (L) ~ theta_5       12
6                                                 MALE_CL/F ~ theta_6       11
7                                                   WT_CL/F ~ theta_7       13
8                                                IIV_CL/F ~ Omega_1.1       10
9                                                cov_CL,V ~ Omega_2.1        8
10                                             IIV_V_c /F ~ Omega_2.2        7
11                                             cov_CL,Ka  ~ Omega_3.1        2
12                                              cov_V,Ka  ~ Omega_3.2        3
13                                               IIV_K_a  ~ Omega_3.3        4
14                                               err_prop ~ Sigma_1.1        5
15                                                err_add ~ Sigma_2.2        9
   prse
1    15
2    14
3    13
4     3
5     5
6     2
7     8
8     6
9     7
10    9
11    4
12   12
13   11
14    1
15   10
\end{Soutput}
\end{Schunk}
Now we can extract some information from the model statements.
\begin{Schunk}
\begin{Sinput}
> tab$units <- justUnits(tab$model)
> tab$model <- noUnits(tab$model)
> tab$name <- with(tab, wiki2label(model))
> tab[c('model','units','name')]
\end{Sinput}
\begin{Soutput}
                                                           model units
1  CL/F  ~ theta_1 *  theta_6 ^MALE * (WT/70)^theta_7  * e^eta_1   L/h
2                        V_c /F  ~ theta_2 * (WT/70)^1 * e^eta_2     L
3                                       K_a  ~ theta_3 * e^eta_3 h^-1 
4                                                 Q/F  ~ theta_4   L/h
5                                              V_p /F  ~ theta_5     L
6                                            MALE_CL/F ~ theta_6      
7                                              WT_CL/F ~ theta_7      
8                                           IIV_CL/F ~ Omega_1.1      
9                                           cov_CL,V ~ Omega_2.1      
10                                        IIV_V_c /F ~ Omega_2.2      
11                                        cov_CL,Ka  ~ Omega_3.1      
12                                         cov_V,Ka  ~ Omega_3.2      
13                                          IIV_K_a  ~ Omega_3.3      
14                                          err_prop ~ Sigma_1.1      
15                                           err_add ~ Sigma_2.2      
        name
1       CL/F
2      V_c/F
3        K_a
4        Q/F
5      V_p/F
6  MALE_CL/F
7    WT_CL/F
8   IIV_CL/F
9   cov_CL,V
10 IIV_V_c/F
11 cov_CL,Ka
12  cov_V,Ka
13   IIV_K_a
14  err_prop
15   err_add
\end{Soutput}
\end{Schunk}
\section{variance}
The estimates for the matrix diagonals are variances, and their square roots have special meaning.  In model 1005, interindividual variability was modelled exponentially, in which case square root of variance gives an approximate CV; alternatively, and exact CV can be calculated.  For proportional error terms like ERR1, square root gives an exact CV.  For additive error terms like ERR2, square root gives standard deviation.

We can use functions of `parameter' to sort out the various error components, as they are used in this model. 
\subsection{exponential}
\begin{Schunk}
\begin{Sinput}
> expo <- is.iiv(tab$parameter) & is.diagonal(tab$parameter)
> tab$parameter[expo]
\end{Sinput}
\begin{Soutput}
[1] "OMEGA1.1" "OMEGA2.2" "OMEGA3.3"
\end{Soutput}
\begin{Sinput}
> tab$cv[expo] <- cvLognormal(tab$estimate[expo])
> tab[,c('parameter','name','estimate','cv')]
\end{Sinput}
\begin{Soutput}
   parameter      name estimate         cv
1     THETA1      CL/F       16         NA
2     THETA2     V_c/F       14         NA
3     THETA3       K_a        6         NA
4     THETA4       Q/F       15         NA
5     THETA5     V_p/F       12         NA
6     THETA6 MALE_CL/F       11         NA
7     THETA7   WT_CL/F       13         NA
8   OMEGA1.1  IIV_CL/F       10 148.409790
9   OMEGA2.1  cov_CL,V        8         NA
10  OMEGA2.2 IIV_V_c/F        7  33.100350
11  OMEGA3.1 cov_CL,Ka        2         NA
12  OMEGA3.2  cov_V,Ka        3         NA
13  OMEGA3.3   IIV_K_a        4   7.321076
14  SIGMA1.1  err_prop        5         NA
15  SIGMA2.2   err_add        9         NA
\end{Soutput}
\end{Schunk}
\subsection{proportional}
\begin{Schunk}
\begin{Sinput}
> writeLines(read.nmctl('../nonmem/ctl/1005.ctl')$err)
\end{Sinput}
\begin{Soutput}
 Y=F*(1+ERR(1)) + ERR(2)
 IPRE=F
;<doc>
\end{Soutput}
\begin{Sinput}
> prop <- is.random(tab$parameter) & tab$name %contains% 'prop'
> tab$parameter[prop]
\end{Sinput}
\begin{Soutput}
[1] "SIGMA1.1"
\end{Soutput}
\begin{Sinput}
> tab$cv[prop] <- sqrt(tab$estimate[prop])
> tab[,c('parameter','name','estimate','cv')]
\end{Sinput}
\begin{Soutput}
   parameter      name estimate         cv
1     THETA1      CL/F       16         NA
2     THETA2     V_c/F       14         NA
3     THETA3       K_a        6         NA
4     THETA4       Q/F       15         NA
5     THETA5     V_p/F       12         NA
6     THETA6 MALE_CL/F       11         NA
7     THETA7   WT_CL/F       13         NA
8   OMEGA1.1  IIV_CL/F       10 148.409790
9   OMEGA2.1  cov_CL,V        8         NA
10  OMEGA2.2 IIV_V_c/F        7  33.100350
11  OMEGA3.1 cov_CL,Ka        2         NA
12  OMEGA3.2  cov_V,Ka        3         NA
13  OMEGA3.3   IIV_K_a        4   7.321076
14  SIGMA1.1  err_prop        5   2.236068
15  SIGMA2.2   err_add        9         NA
\end{Soutput}
\end{Schunk}
\subsection{additive}
\begin{Schunk}
\begin{Sinput}
> add <- is.residual(tab$parameter) & tab$name %contains% 'add'
> tab$parameter[add]
\end{Sinput}
\begin{Soutput}
[1] "SIGMA2.2"
\end{Soutput}
\begin{Sinput}
> tab$sd[add] <- sqrt(tab$estimate[add])
> tab[,c('parameter','name','estimate','cv','sd')]
\end{Sinput}
\begin{Soutput}
   parameter      name estimate         cv sd
1     THETA1      CL/F       16         NA NA
2     THETA2     V_c/F       14         NA NA
3     THETA3       K_a        6         NA NA
4     THETA4       Q/F       15         NA NA
5     THETA5     V_p/F       12         NA NA
6     THETA6 MALE_CL/F       11         NA NA
7     THETA7   WT_CL/F       13         NA NA
8   OMEGA1.1  IIV_CL/F       10 148.409790 NA
9   OMEGA2.1  cov_CL,V        8         NA NA
10  OMEGA2.2 IIV_V_c/F        7  33.100350 NA
11  OMEGA3.1 cov_CL,Ka        2         NA NA
12  OMEGA3.2  cov_V,Ka        3         NA NA
13  OMEGA3.3   IIV_K_a        4   7.321076 NA
14  SIGMA1.1  err_prop        5   2.236068 NA
15  SIGMA2.2   err_add        9         NA  3
\end{Soutput}
\end{Schunk}
\section{covariance}
The estimates of matrix off-diagonals are covariances, and are more useful if transformed to correlations.  We could extract the matrices manually, or use package shortcuts.
\begin{Schunk}
\begin{Sinput}
> cor <- omegacor(run=1005,project='../nonmem')
> cor
\end{Sinput}
\begin{Soutput}
           [,1]       [,2]       [,3]
[1,]  1.0000000  0.8494277 -0.1162464
[2,]  0.8494277  1.0000000 -0.5605290
[3,] -0.1162464 -0.5605290  1.0000000
\end{Soutput}
\begin{Sinput}
> half(cor)
\end{Sinput}
\begin{Soutput}
       1.1        2.1        2.2        3.1        3.2        3.3 
 1.0000000  0.8494277  1.0000000 -0.1162464 -0.5605290  1.0000000 
\end{Soutput}
\begin{Sinput}
> offdiag(half(cor))
\end{Sinput}
\begin{Soutput}
       2.1        3.1        3.2 
 0.8494277 -0.1162464 -0.5605290 
\end{Soutput}
\begin{Sinput}
> off <- is.iiv(tab$parameter) & is.offdiagonal(tab$parameter)
> tab$parameter[off]
\end{Sinput}
\begin{Soutput}
[1] "OMEGA2.1" "OMEGA3.1" "OMEGA3.2"
\end{Soutput}
\begin{Sinput}
> tab$cor[off] <- offdiag(half(cor))
> tab[,c('parameter','name','estimate','cv','sd','cor')]
\end{Sinput}
\begin{Soutput}
   parameter      name estimate         cv sd        cor
1     THETA1      CL/F       16         NA NA         NA
2     THETA2     V_c/F       14         NA NA         NA
3     THETA3       K_a        6         NA NA         NA
4     THETA4       Q/F       15         NA NA         NA
5     THETA5     V_p/F       12         NA NA         NA
6     THETA6 MALE_CL/F       11         NA NA         NA
7     THETA7   WT_CL/F       13         NA NA         NA
8   OMEGA1.1  IIV_CL/F       10 148.409790 NA         NA
9   OMEGA2.1  cov_CL,V        8         NA NA  0.8494277
10  OMEGA2.2 IIV_V_c/F        7  33.100350 NA         NA
11  OMEGA3.1 cov_CL,Ka        2         NA NA -0.1162464
12  OMEGA3.2  cov_V,Ka        3         NA NA -0.5605290
13  OMEGA3.3   IIV_K_a        4   7.321076 NA         NA
14  SIGMA1.1  err_prop        5   2.236068 NA         NA
15  SIGMA2.2   err_add        9         NA  3         NA
\end{Soutput}
\end{Schunk}
\section{confidence interval}
We wish to include 95 percentiles in our table as confidence intervals.
\begin{Schunk}
\begin{Sinput}
> boot <- read.csv('../nonmem/1005bootlog.csv',as.is=TRUE)
> head(boot)
\end{Sinput}
\begin{Soutput}
  X tool run parameter   moment           value
1 1  nm7   1       ofv  minimum 2641.7825682304
2 2  nm7   1    THETA1 estimate         9.23638
3 3  nm7   1    THETA1     prse            <NA>
4 4  nm7   1    THETA1       se            <NA>
5 5  nm7   1    THETA2 estimate         23.3418
6 6  nm7   1    THETA2     prse            <NA>
\end{Soutput}
\begin{Sinput}
> boot <- boot[boot$moment=='estimate',]
> boot <- data.frame(cast(boot,...~moment))
> head(boot)
\end{Sinput}
\begin{Soutput}
   X tool run parameter  estimate
1  2  nm7   1    THETA1   9.23638
2  5  nm7   1    THETA2   23.3418
3  8  nm7   1    THETA3 0.0677011
4 11  nm7   1    THETA4   3.82773
5 14  nm7   1    THETA5    114.89
6 17  nm7   1    THETA6  0.981208
\end{Soutput}
\begin{Sinput}
> boot <- boot[,c('run','parameter','estimate')]
> sapply(boot,class)
\end{Sinput}
\begin{Soutput}
        run   parameter    estimate 
  "integer" "character"    "factor" 
\end{Soutput}
\begin{Sinput}
> boot$estimate <- as.numeric(as.character(boot$estimate))
> unique(boot$parameter)
\end{Sinput}
\begin{Soutput}
 [1] "THETA1"   "THETA2"   "THETA3"   "THETA4"   "THETA5"   "THETA6"  
 [7] "THETA7"   "OMEGA1.1" "OMEGA2.1" "OMEGA2.2" "OMEGA3.1" "OMEGA3.2"
[13] "OMEGA3.3" "SIGMA1.1" "SIGMA2.1" "SIGMA2.2"
\end{Soutput}
\begin{Sinput}
> quan <- function(x,probs)as.character(signif(quantile(x,probs=probs,na.rm=TRUE),3))
> boot$lo <- with(boot, reapply(estimate,parameter,quan,probs=.05))
> boot$hi <- with(boot, reapply(estimate,parameter,quan,probs=.95))
> head(boot)
\end{Sinput}
\begin{Soutput}
  run parameter    estimate     lo     hi
1   1    THETA1   9.2363800   6.67   11.1
2   1    THETA2  23.3418000     19   27.6
3   1    THETA3   0.0677011 0.0636 0.0814
4   1    THETA4   3.8277300   2.77   4.97
5   1    THETA5 114.8900000   87.9    315
6   1    THETA6   0.9812080  0.845   1.27
\end{Soutput}
\begin{Sinput}
> boot <- unique(boot[,c('parameter','lo','hi')])
> boot
\end{Sinput}
\begin{Soutput}
   parameter      lo       hi
1     THETA1    6.67     11.1
2     THETA2      19     27.6
3     THETA3  0.0636   0.0814
4     THETA4    2.77     4.97
5     THETA5    87.9      315
6     THETA6   0.845     1.27
7     THETA7   0.685     1.96
8   OMEGA1.1   0.127    0.325
9   OMEGA2.1  0.0657    0.183
10  OMEGA2.2  0.0457    0.159
11  OMEGA3.1 -0.0438   0.0248
12  OMEGA3.2 -0.0536 -0.00759
13  OMEGA3.3  0.0237   0.0789
14  SIGMA1.1  0.0404   0.0594
15  SIGMA2.1       0        0
16  SIGMA2.2   0.073    0.323
\end{Soutput}
\begin{Sinput}
> boot$ci <- with(boot, parens(glue(lo,',',hi)))
> boot
\end{Sinput}
\begin{Soutput}
   parameter      lo       hi                 ci
1     THETA1    6.67     11.1        (6.67,11.1)
2     THETA2      19     27.6          (19,27.6)
3     THETA3  0.0636   0.0814    (0.0636,0.0814)
4     THETA4    2.77     4.97        (2.77,4.97)
5     THETA5    87.9      315         (87.9,315)
6     THETA6   0.845     1.27       (0.845,1.27)
7     THETA7   0.685     1.96       (0.685,1.96)
8   OMEGA1.1   0.127    0.325      (0.127,0.325)
9   OMEGA2.1  0.0657    0.183     (0.0657,0.183)
10  OMEGA2.2  0.0457    0.159     (0.0457,0.159)
11  OMEGA3.1 -0.0438   0.0248   (-0.0438,0.0248)
12  OMEGA3.2 -0.0536 -0.00759 (-0.0536,-0.00759)
13  OMEGA3.3  0.0237   0.0789    (0.0237,0.0789)
14  SIGMA1.1  0.0404   0.0594    (0.0404,0.0594)
15  SIGMA2.1       0        0              (0,0)
16  SIGMA2.2   0.073    0.323      (0.073,0.323)
\end{Soutput}
\begin{Sinput}
> tab <- stableMerge(tab,boot[,c('parameter','ci')])
> tab
\end{Sinput}
\begin{Soutput}
   parameter                                   description
1     THETA1                       apparent oral clearance
2     THETA2                central volume of distribution
3     THETA3                      absorption rate constant
4     THETA4                  intercompartmental clearance
5     THETA5             peripheral volume of distribution
6     THETA6                      male effect on clearance
7     THETA7                    weight effect on clearance
8   OMEGA1.1      interindividual variability of clearance
9   OMEGA2.1   interindividual clearance-volume covariance
10  OMEGA2.2 interindividual variability of central volume
11  OMEGA3.1       interindividual clearance-Ka covariance
12  OMEGA3.2          interindividual volume-Ka covariance
13  OMEGA3.3             interindividual variability of Ka
14  SIGMA1.1                            proportional error
15  SIGMA2.2                                additive error
                                                           model estimate prse
1  CL/F  ~ theta_1 *  theta_6 ^MALE * (WT/70)^theta_7  * e^eta_1       16   15
2                        V_c /F  ~ theta_2 * (WT/70)^1 * e^eta_2       14   14
3                                       K_a  ~ theta_3 * e^eta_3        6   13
4                                                 Q/F  ~ theta_4       15    3
5                                              V_p /F  ~ theta_5       12    5
6                                            MALE_CL/F ~ theta_6       11    2
7                                              WT_CL/F ~ theta_7       13    8
8                                           IIV_CL/F ~ Omega_1.1       10    6
9                                           cov_CL,V ~ Omega_2.1        8    7
10                                        IIV_V_c /F ~ Omega_2.2        7    9
11                                        cov_CL,Ka  ~ Omega_3.1        2    4
12                                         cov_V,Ka  ~ Omega_3.2        3   12
13                                          IIV_K_a  ~ Omega_3.3        4   11
14                                          err_prop ~ Sigma_1.1        5    1
15                                           err_add ~ Sigma_2.2        9   10
   units      name         cv sd        cor                 ci
1    L/h      CL/F         NA NA         NA        (6.67,11.1)
2      L     V_c/F         NA NA         NA          (19,27.6)
3  h^-1        K_a         NA NA         NA    (0.0636,0.0814)
4    L/h       Q/F         NA NA         NA        (2.77,4.97)
5      L     V_p/F         NA NA         NA         (87.9,315)
6        MALE_CL/F         NA NA         NA       (0.845,1.27)
7          WT_CL/F         NA NA         NA       (0.685,1.96)
8         IIV_CL/F 148.409790 NA         NA      (0.127,0.325)
9         cov_CL,V         NA NA  0.8494277     (0.0657,0.183)
10       IIV_V_c/F  33.100350 NA         NA     (0.0457,0.159)
11       cov_CL,Ka         NA NA -0.1162464   (-0.0438,0.0248)
12        cov_V,Ka         NA NA -0.5605290 (-0.0536,-0.00759)
13         IIV_K_a   7.321076 NA         NA    (0.0237,0.0789)
14        err_prop   2.236068 NA         NA    (0.0404,0.0594)
15         err_add         NA  3         NA      (0.073,0.323)
\end{Soutput}
\end{Schunk}
\section{aesthetics}
Here we format the table for printing.
\begin{Schunk}
\begin{Sinput}
> tab$name <- NULL
> tab$parameter <- NULL
> tab$model <- wiki2latex(tab$model)
> tab$estimate <- as.character(tab$estimate)
> tab$estimate <- paste(tab$estimate,'$', tab$units,'$')
> tab$units <- NULL
\end{Sinput}
\end{Schunk}
Note that no parameter defines more than one of CV, SD,and COR.  We could collapse these into a single column, and add a descriptive flag.
\begin{Schunk}
\begin{Sinput}
> m <- as.matrix(tab[,c('cv','sd','cor')])
> tab$variability <- suppressWarnings(apply(m,1,max,na.rm=TRUE))
> tab$variability[is.infinite(tab$variability)] <- NA
> i <- is.defined(m)
> i[!i] <- NA
> tab$statistic <- apply(i,1,function(x){
+   p <- colnames(i)[x]
+   ifelse(all(is.na(p)),NA,p[!is.na(p)])
+ })
> toPercent <- with(tab, !is.na(statistic) & statistic=='cv')
> tab$variability[toPercent] <- percent(tab$variability[toPercent])
> tab$variability <- as.character(signif(tab$variability,3))
> tab$statistic <- map(tab$statistic,from=c(NA,'cv','cor','sd'),to=c(NA,'\\%CV','CORR','SD'))
> tab$variability <- paste(tab$statistic,tab$variability,sep=' = ')
> tab$variability[is.na(tab$statistic)] <- NA
> tab$statistic <- NULL
> tab$cv <- NULL
> tab$sd <- NULL
> tab$cor <- NULL
\end{Sinput}
\end{Schunk}
\begin{table}[!htpb]
 \caption[Model 1005 Parameters]{Parameter Estimates from Population Pharmacokinetic Model Run 1005 \label{p1005}}
 \begin{center}
  \begin{tabular}{llllll}
    \hline \hline
   description & model & estimate & prse & ci & variability \\ \hline
   apparent oral clearance                       & $\mathrm{CL/F  \sim \theta_{1}\cdot  \theta_{6}^{MALE}\cdot (WT/70)^{\theta_{7}}\cdot e^{\eta_{1}}}$ & 16 $ L/h $  & 15 & (6.67,11.1)        &  \\
   central volume of distribution                & $\mathrm{V_{c}/F  \sim \theta_{2}\cdot (WT/70)^{1}\cdot e^{\eta_{2}}}$                                  & 14 $ L $    & 14 & (19,27.6)          &  \\
   absorption rate constant                      & $\mathrm{K_{a} \sim \theta_{3}\cdot e^{\eta_{3}}}$                                                       & 6 $ h^-1  $ & 13 & (0.0636,0.0814)    &  \\
   intercompartmental clearance                  & $\mathrm{Q/F  \sim \theta_{4}}$                                                                            & 15 $ L/h $  & 3  & (2.77,4.97)        &  \\
   peripheral volume of distribution             & $\mathrm{V_{p}/F  \sim \theta_{5}}$                                                                        & 12 $ L $    & 5  & (87.9,315)         &  \\
   male effect on clearance                      & $\mathrm{MALE_{CL/F}\sim \theta_{6}}$                                                                      & 11 $  $     & 2  & (0.845,1.27)       &  \\
   weight effect on clearance                    & $\mathrm{WT_{CL/F}\sim \theta_{7}}$                                                                        & 13 $  $     & 8  & (0.685,1.96)       &  \\
   interindividual variability of clearance      & $\mathrm{IIV_{CL/F}\sim \Omega_{1.1}}$                                                                     & 10 $  $     & 6  & (0.127,0.325)      & \%CV = 14800 \\
   interindividual clearance-volume covariance   & $\mathrm{cov_{CL,V}\sim \Omega_{2.1}}$                                                                     & 8 $  $      & 7  & (0.0657,0.183)     & CORR = 0.849  \\
   interindividual variability of central volume & $\mathrm{IIV_{V_{c}/F}\sim \Omega_{2.2}}$                                                                  & 7 $  $      & 9  & (0.0457,0.159)     & \%CV = 3310  \\
   interindividual clearance-Ka covariance       & $\mathrm{cov_{CL,Ka} \sim \Omega_{3.1}}$                                                                   & 2 $  $      & 4  & (-0.0438,0.0248)   & CORR = -0.116 \\
   interindividual volume-Ka covariance          & $\mathrm{cov_{V,Ka} \sim \Omega_{3.2}}$                                                                    & 3 $  $      & 12 & (-0.0536,-0.00759) & CORR = -0.561 \\
   interindividual variability of Ka             & $\mathrm{IIV_{K_{a}}\sim \Omega_{3.3}}$                                                                    & 4 $  $      & 11 & (0.0237,0.0789)    & \%CV = 732   \\
   proportional error                            & $\mathrm{err_{prop}\sim \Sigma_{1.1}}$                                                                     & 5 $  $      & 1  & (0.0404,0.0594)    & \%CV = 224   \\
   additive error                                & $\mathrm{err_{add}\sim \Sigma_{2.2}}$                                                                      & 9 $  $      & 10 & (0.073,0.323)      & SD = 3        \\ \hline
  \end{tabular}
 \end{center}
\end{table}\section{simple parameter table}
We can make a quick parameter table that does not use wikitab markup. Table \ref{simple}.
\begin{Schunk}
\begin{Sinput}
> tab <- rlog(1005,'../nonmem',tool='nm7',file=NULL)
> head(tab)
\end{Sinput}
\begin{Soutput}
  tool  run parameter   moment            value
1  nm7 1005       ofv  minimum 2405.91626347113
2  nm7 1005    THETA1 estimate           9.5079
3  nm7 1005    THETA1     prse             9.72
4  nm7 1005    THETA1       se         0.923845
5  nm7 1005    THETA2 estimate          22.7899
6  nm7 1005    THETA2     prse             9.56
\end{Soutput}
\begin{Sinput}
> tab$tool <- NULL
> tab$run <- NULL
> tab <- tab[tab$moment %in% c('estimate','prse'),]
> unique(tab$parameter)
\end{Sinput}
\begin{Soutput}
 [1] "THETA1"   "THETA2"   "THETA3"   "THETA4"   "THETA5"   "THETA6"  
 [7] "THETA7"   "OMEGA1.1" "OMEGA2.1" "OMEGA2.2" "OMEGA3.1" "OMEGA3.2"
[13] "OMEGA3.3" "SIGMA1.1" "SIGMA2.1" "SIGMA2.2"
\end{Soutput}
\begin{Sinput}
> tab$value <- signif(as.numeric(tab$value),3)
> tab$parameter <- factor(tab$parameter,levels=unique(tab$parameter))#to preserve row order during cast
> tab <- cast(tab,parameter~moment)
> tab
\end{Sinput}
\begin{Soutput}
   parameter estimate   prse
1     THETA1   9.5100   9.72
2     THETA2  22.8000   9.56
3     THETA3   0.0714   7.34
4     THETA4   3.4700  15.40
5     THETA5 113.0000  20.90
6     THETA6   1.0200  11.00
7     THETA7   1.1900  28.30
8   OMEGA1.1   0.2140  22.80
9   OMEGA2.1   0.1210  26.40
10  OMEGA2.2   0.0945  33.20
11  OMEGA3.1  -0.0116 173.00
12  OMEGA3.2  -0.0372  36.10
13  OMEGA3.3   0.0466  34.80
14  SIGMA1.1   0.0492  10.90
15  SIGMA2.1   0.0000    Inf
16  SIGMA2.2   0.2020  33.50
\end{Soutput}
\begin{Sinput}
> tab$parameter <- parameter2wiki(tab$parameter)
> tab
\end{Sinput}
\begin{Soutput}
    parameter estimate   prse
1    theta_1    9.5100   9.72
2    theta_2   22.8000   9.56
3    theta_3    0.0714   7.34
4    theta_4    3.4700  15.40
5    theta_5  113.0000  20.90
6    theta_6    1.0200  11.00
7    theta_7    1.1900  28.30
8  Omega_1.1    0.2140  22.80
9  Omega_2.1    0.1210  26.40
10 Omega_2.2    0.0945  33.20
11 Omega_3.1   -0.0116 173.00
12 Omega_3.2   -0.0372  36.10
13 Omega_3.3    0.0466  34.80
14 Sigma_1.1    0.0492  10.90
15 Sigma_2.1    0.0000    Inf
16 Sigma_2.2    0.2020  33.50
\end{Soutput}
\begin{Sinput}
> tab$parameter <- wiki2latex(tab$parameter)
> tab
\end{Sinput}
\begin{Soutput}
                   parameter estimate   prse
1    $\\mathrm{\\theta_{1}}$   9.5100   9.72
2    $\\mathrm{\\theta_{2}}$  22.8000   9.56
3    $\\mathrm{\\theta_{3}}$   0.0714   7.34
4    $\\mathrm{\\theta_{4}}$   3.4700  15.40
5    $\\mathrm{\\theta_{5}}$ 113.0000  20.90
6    $\\mathrm{\\theta_{6}}$   1.0200  11.00
7    $\\mathrm{\\theta_{7}}$   1.1900  28.30
8  $\\mathrm{\\Omega_{1.1}}$   0.2140  22.80
9  $\\mathrm{\\Omega_{2.1}}$   0.1210  26.40
10 $\\mathrm{\\Omega_{2.2}}$   0.0945  33.20
11 $\\mathrm{\\Omega_{3.1}}$  -0.0116 173.00
12 $\\mathrm{\\Omega_{3.2}}$  -0.0372  36.10
13 $\\mathrm{\\Omega_{3.3}}$   0.0466  34.80
14 $\\mathrm{\\Sigma_{1.1}}$   0.0492  10.90
15 $\\mathrm{\\Sigma_{2.1}}$   0.0000    Inf
16 $\\mathrm{\\Sigma_{2.2}}$   0.2020  33.50
\end{Soutput}
\end{Schunk}
\begin{table}[!htpb]
 \caption[Simple Parameter Table]{Simple Parameter Table \label{simple}}
 \begin{center}
  \begin{tabular}{lrr}
    \hline \hline
   parameter & estimate & prse \\ \hline
   $\mathrm{\theta_{1}}$   & \verb#9.5100# & \verb#9.72# \\
   $\mathrm{\theta_{2}}$   & \verb#22.8000# & \verb#9.56# \\
   $\mathrm{\theta_{3}}$   & \verb#0.0714# & \verb#7.34# \\
   $\mathrm{\theta_{4}}$   & \verb#3.4700# & \verb#15.40# \\
   $\mathrm{\theta_{5}}$   & \verb#113.0000# & \verb#20.90# \\
   $\mathrm{\theta_{6}}$   & \verb#1.0200# & \verb#11.00# \\
   $\mathrm{\theta_{7}}$   & \verb#1.1900# & \verb#28.30# \\
   $\mathrm{\Omega_{1.1}}$ & \verb#0.2140# & \verb#22.80# \\
   $\mathrm{\Omega_{2.1}}$ & \verb#0.1210# & \verb#26.40# \\
   $\mathrm{\Omega_{2.2}}$ & \verb#0.0945# & \verb#33.20# \\
   $\mathrm{\Omega_{3.1}}$ & \verb#-0.0116# & \verb#173.00# \\
   $\mathrm{\Omega_{3.2}}$ & \verb#-0.0372# & \verb#36.10# \\
   $\mathrm{\Omega_{3.3}}$ & \verb#0.0466# & \verb#34.80# \\
   $\mathrm{\Sigma_{1.1}}$ & \verb#0.0492# & \verb#10.90# \\
   $\mathrm{\Sigma_{2.1}}$ & \verb#0.0000# & \verb#Inf# \\
   $\mathrm{\Sigma_{2.2}}$ & \verb#0.2020# & \verb#33.50# \\ \hline
  \end{tabular}
 \end{center}
\end{table}\end{document}
