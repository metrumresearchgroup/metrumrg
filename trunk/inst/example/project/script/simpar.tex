\documentclass[titlepage]{article}
\usepackage{listings}
\lstMakeShortInline{|}
\usepackage{courier}
%\usepackage{hyperref}
\usepackage[colorlinks,linkcolor=blue,citecolor=blue,urlcolor=blue,breaklinks=true]{hyperref}
\lstset{basicstyle=\ttfamily\small , breaklines}
%\usepackage[margin=2cm]{geometry}
\usepackage[left=3cm,top=3cm,bottom=3cm, right=3cm,includehead,includefoot,landscape]{geometry}
\usepackage{color}
\usepackage{fancyhdr,lastpage}
\pagestyle{fancy}
\rhead{Metrum Research Group, LLC \\ }
\lhead{\includegraphics[scale=2]{logo.png}}
\cfoot{Page \thepage\ of \pageref{LastPage}}
\fancyhfoffset{.25in}
\renewcommand{\headrulewidth}{0.25pt}
\renewcommand{\footrulewidth}{0pt} 
\setlength{\headheight}{23pt}
\renewcommand{\labelitemiii}{$\circ$}
\usepackage{longtable}
\usepackage{amsmath}
\usepackage[T1]{fontenc}
\usepackage[scaled]{helvet}
\renewcommand*\familydefault{\sfdefault}
\usepackage{courier}
\usepackage{graphicx}
\usepackage{tocbibind}
\usepackage[parfill]{parskip}    % Activate to begin paragraphs with an empty line rather than an indent
\usepackage{upgreek}
\usepackage{textpos}
\usepackage{relsize}
\usepackage{upquote}
% Use \begin{landscape} and end{landscape} to rotate text %%%
\usepackage{pdflscape}
\usepackage{textcomp}
\usepackage{float}
\floatplacement{figure}{H}
\floatplacement{table}{H}
\usepackage[printonlyused,nohyperlinks]{acronym}
\def\bflabel#1{{\large#1\ \ \ \ }\hfill}
\usepackage{fixltx2e}
\setlength{\belowcaptionskip}{10pt}





\usepackage{Sweave}

 
\begin{document}
\vspace*{2cm}
\begin{center}
{\Large Simulating with Parameter Uncertainty}\\
~\\
\today\\
~\\
Bill Knebel\\
Tim Bergsma\\
\end{center}
\newpage

\section{Purpose}
This script shows how to conduct a simulation that
considers uncertainty in the parameter estimates.
\section{Data}
Here we load metrumrg and read in the data to be used
for simulations.
\begin{Schunk}
\begin{Sinput}
> library(metrumrg)
> data <- read.csv("../data/derived/phase1.csv")
> head(data)
\end{Sinput}
\begin{Soutput}
  C ID TIME SEQ EVID  AMT    DV SUBJ HOUR TAFD  TAD LDOS MDV HEIGHT WEIGHT SEX
1 C  1 0.00   0    0    .     0    1 0.00 0.00    .    .   0    174   74.2   0
2 .  1 0.00   1    1 1000     .    1 0.00 0.00    0 1000   1    174   74.2   0
3 .  1 0.25   0    0    . 0.363    1 0.25 0.25 0.25 1000   0    174   74.2   0
4 .  1 0.50   0    0    . 0.914    1 0.50 0.50  0.5 1000   0    174   74.2   0
5 .  1 1.00   0    0    .  1.12    1 1.00 1.00    1 1000   0    174   74.2   0
6 .  1 2.00   0    0    .  2.28    1 2.00 2.00    2 1000   0    174   74.2   0
   AGE DOSE FED SMK DS CRCN predose zerodv
1 29.1 1000   1   0  0 83.5       1      0
2 29.1 1000   1   0  0 83.5       0      0
3 29.1 1000   1   0  0 83.5       0      0
4 29.1 1000   1   0  0 83.5       0      0
5 29.1 1000   1   0  0 83.5       0      0
6 29.1 1000   1   0  0 83.5       0      0
\end{Soutput}
\end{Schunk}
We use NONMEM output from a simple two compartment model to generate parameters.
We use 1005.lst and 1005.cov output from NM7 to populate a call to metrumrg::simpar().
\begin{Schunk}
\begin{Sinput}
> cov <- read.table("../nonmem/1005/1005.cov", skip=1, header=T)
> head(cov)
\end{Sinput}
\begin{Soutput}
    NAME     THETA1      THETA2       THETA3      THETA4      THETA5
1 THETA1  0.8759870  0.79288000  1.06070e-03  0.06301820  -1.7590100
2 THETA2  0.7928800  4.74648000  6.68044e-03  0.89631100   5.2850000
3 THETA3  0.0010607  0.00668044  2.75847e-05  0.00222162  -0.0305675
4 THETA4  0.0630182  0.89631100  2.22162e-03  0.28693100   0.1902840
5 THETA5 -1.7590100  5.28500000 -3.05675e-02  0.19028400 566.1060000
6 THETA6 -0.0421234 -0.02569100 -1.05405e-04 -0.01064250   0.7806910
        THETA6       THETA7   SIGMA.1.1. SIGMA.2.1.   SIGMA.2.2.  OMEGA.1.1.
1 -0.042123400 -0.180383000 -5.30926e-04          0  2.09474e-02 6.20306e-03
2 -0.025691000  0.066978500 -3.13015e-03          0  1.92041e-02 5.76993e-03
3 -0.000105405 -0.000138306 -1.02726e-05          0  5.91125e-05 3.21603e-06
4 -0.010642500  0.015518300 -6.29448e-04          0  2.54028e-03 4.29694e-03
5  0.780691000 -0.602816000  4.56339e-02          0 -4.26533e-01 2.73809e-01
6  0.013065500  0.000798277  1.21261e-04          0 -1.07146e-03 1.64777e-03
    OMEGA.2.1.   OMEGA.2.2.   OMEGA.3.1.   OMEGA.3.2.   OMEGA.3.3.
1 -1.83821e-04 -4.40561e-03 -5.44206e-03 -2.59575e-03 -3.35118e-03
2 -2.19650e-02 -2.44862e-02 -1.96039e-02 -1.12215e-02  4.77949e-03
3 -6.50573e-05 -7.81727e-05 -6.76632e-05 -2.75979e-05  2.83104e-05
4 -6.21366e-03 -7.78974e-03 -4.55505e-03 -2.25175e-03  3.07354e-03
5  1.60570e-01  2.81746e-02 -4.85574e-03  7.48252e-02 -3.43426e-02
6  3.02269e-04  6.01990e-04 -5.29743e-04 -5.31172e-05 -3.35780e-04
\end{Soutput}
\end{Schunk}
We are interested in theta covariance, so we remove extra columns and rows.
\begin{Schunk}
\begin{Sinput}
> cov<- cov[1:7,c(2:8)]
\end{Sinput}
\end{Schunk}
\section{Parameters}
Now we generate 10 sets of population parameters based on the 1005.lst results.
\begin{Schunk}
\begin{Sinput}
> set.seed(10)
> PKparms <- simpar(
+     nsim=10,
+     theta=c(8.58,21.6, 0.0684, 3.78, 107, 0.999, 1.67),
+     covar=cov,
+     omega=list(0.196, 0.129, 0.107),
+     odf=c(40,40,40),
+     sigma=list(0.0671),
+     sdf=c(200)
+ )
> PKparms
\end{Sinput}
\begin{Soutput}
    TH.1  TH.2    TH.3  TH.4   TH.5   TH.6  TH.7  OM1.1   OM2.2   OM3.3   SG1.1
1  7.554 19.23 0.06670 3.883 107.50 1.1030 1.343 0.1847 0.15400 0.13630 0.06894
2  6.513 20.19 0.06640 3.864 102.60 1.0690 2.331 0.2862 0.12000 0.16400 0.06099
3  8.261 21.93 0.06600 3.723  74.37 0.8289 2.140 0.1647 0.12770 0.11300 0.06041
4  6.377 19.66 0.06681 3.524  92.75 0.9410 2.017 0.1886 0.11460 0.08460 0.07700
5  7.252 20.13 0.07282 4.138 114.00 0.9482 1.942 0.1526 0.08448 0.13140 0.06269
6  8.200 21.47 0.07480 4.221 116.30 0.9346 1.545 0.2462 0.17640 0.08805 0.07274
7  8.502 23.51 0.07477 4.147  78.24 1.0610 1.905 0.2221 0.14440 0.09957 0.06160
8  7.984 21.95 0.07319 4.525  98.34 0.9233 1.701 0.2287 0.13820 0.06118 0.06692
9  8.269 19.21 0.07019 3.555  68.31 0.9783 1.813 0.1765 0.12310 0.08504 0.06092
10 8.140 20.52 0.06545 3.755 100.90 1.0090 1.512 0.2116 0.11940 0.09954 0.06269
\end{Soutput}
\end{Schunk}
\section{Control Streams}
We read in a control stream and clean out extra xml markup.
\begin{Schunk}
\begin{Sinput}
> ctl <- as.nmctl(readLines("../nonmem/ctl/1005.ctl"))
> ctl[] <- lapply(ctl,function(rec)sub("<.*","",rec))
\end{Sinput}
\end{Schunk}
Now we iterate across the rows of PKparms, writing out a separate ctl for each.
\begin{Schunk}
\begin{Sinput}
> dir.create('../nonmem/sim')
> set <- lapply(
+ 	rownames(PKparms),
+ 	function(row,params,ctl){
+ 		params <- as.character(PKparms[row,])
+ 		ctl$prob <- sub(1005,row,ctl$prob)
+ 		ctl$theta <- params[1:7]
+ 		ctl$omega <- params[8:10]
+ 		ctl$sigma <- params[11]
+ 		names(ctl)[names(ctl)=='estimation'] <- 'simulation'
+ 		ctl$simulation <- paste(
+ 			'(',
+ 			as.numeric(row) + 7995,
+ 			'NEW) (',
+ 			as.numeric(row) + 8996,
+ 			'UNIFORM) ONLYSIMULATION'
+ 		)
+ 		ctl$cov <- NULL
+ 		ctl$table <- NULL
+ 		ctl$table <- NULL
+ 		ctl$table <- 'ID TIME DV WT SEX LDOS NOPRINT NOAPPEND FILE=sim.tab'
+ 		write.nmctl(ctl,file=file.path('../nonmem/sim',paste(sep='.',row,'ctl')))
+ 		return(ctl)		
+ 	},
+ 	params=PKparms,
+ 	ctl=ctl
+ )
\end{Sinput}
\end{Schunk}
\section{Simulation}
Finally, we run NONMEM simulations using NONR.
\begin{Schunk}
\begin{Sinput}
> NONR72(
+ 	run=1:10,
+ 	command="/opt/NONMEM/nm72/nmqual/autolog.pl",
+ 	project="../nonmem/sim",
+ 	diag=FALSE,
+ 	checkrunno=FALSE,
+ 	grid=TRUE
+ )
\end{Sinput}
\end{Schunk}
\end{document}
